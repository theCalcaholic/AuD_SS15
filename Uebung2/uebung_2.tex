\documentclass[fleqn]{article}
\usepackage[utf8]{inputenc}
\usepackage{geometry}
\usepackage{fancyhdr}
\usepackage{amsmath,amsthm,amssymb}
\usepackage{graphicx}
\usepackage{hyperref}
\usepackage{lipsum}
\usepackage{ulem}
\usepackage{comment}
\usepackage{enumerate}
\usepackage{titlesec}
\usepackage{listings}

\lstset{
  basicstyle=\scriptsize\ttfamily,
  keywordstyle=\bfseries\ttfamily\color{orange},
  stringstyle=\color{green}\ttfamily,
  commentstyle=\color{middlegray}\ttfamily,
  emph={square}, 
  emphstyle=\color{blue}\texttt,
  emph={[2]root,base},
  emphstyle={[2]\color{yac}\texttt},
  showstringspaces=false,
  flexiblecolumns=false,
  tabsize=2,
  numbers=left,
  numberstyle=\tiny,
  numberblanklines=false,
  stepnumber=1,
  numbersep=6pt,
  xleftmargin=15pt
}

\date{\gertoday}

\lhead{\@author}
\chead{}
\rhead{\gertoday}
\lfoot{}
\cfoot{\thepage}
\rfoot{}
\setlength{\mathindent}{0pt}

\usepackage[x11names, rgb]{xcolor}
\usepackage[utf8]{inputenc}
\usepackage{tikz}
\usepackage{geometry}
\usepackage{fancyhdr}
\usepackage{amsmath,amsthm,amssymb}
\usepackage{graphicx}
\usepackage{hyperref}
\usepackage{lipsum}
\usepackage{ulem}
\usepackage{comment}
\usepackage{enumerate}
\usepackage{titlesec}
\usepackage{boolexpr,pdftexcmds,trace}
\makeatletter

\usetikzlibrary{snakes,arrows,shapes}
\newwrite\dotfile

\begingroup
  \catcode`\[ = 1\relax
  \catcode`\] = 2\relax
  \catcode`\{ = 12\relax
  \catcode`\} = 12 \relax
  \gdef\OpenBrace[{]
  \gdef\CloseBrace[}]
\endgroup

% custom commands
\newcommand{\leadingzero}[1]{\ifnum #1<10 0#1\else#1\fi}
\newcommand{\gerdate}[3]{\leadingzero{#1}.\leadingzero{#2}.\leadingzero{#3}}
\newcommand{\gertoday}{\gerdate{\the\day}{\the\month}{\the\year}}
\newcommand*{\bfrac}[2]{\genfrac{}{}{0pt}{}{#1}{#2}}
\newcommand{\R}{\mathbb{R}}
\newcommand{\N}{\mathbb{N}}
\newcommand{\Q}{\mathbb{Q}}
\newcommand{\dotarrow}[0]{}

\newenvironment{graphviz}[1]%
{%
\switch
\case{\pdf@strcmp{#1}{graph}}
    \renewcommand{\dotarrow}[0]{--}
\case{\pdf@strcmp{#1}{strict graph}}
    \renewcommand{\dotarrow}[0]{--}
\case{\pdf@strcmp{#1}{digraph}}
    \renewcommand{\dotarrow}[0]{->}
\case{\pdf@strcmp{#1}{strict digraph}}
    \renewcommand{\dotarrow}[0]{->}
\endswitch

\immediate\openout\dotfile=tmp.dot%
\newcommand{\node}[2]{%
\immediate\write\dotfile{##1 \dotarrow \OpenBrace##2\CloseBrace}%
}%
%
\immediate\write\dotfile{#1 \OpenBrace}
}%
{\immediate\write\dotfile{\CloseBrace}%
\immediate\closeout\dotfile%
\immediate\write18{dot2tex --figonly tmp.dot > tmp.tex}%
\input{tmp.tex}%
}

\setcounter{section}{0}
\setcounter{subsection}{0}
\pagestyle{fancy}

\lhead{}
\chead{}
\rhead{\gertoday}
\lfoot{}
\cfoot{\thepage}
\rfoot{}
\setlength{\mathindent}{0pt}

% document specific settings and formats
% set consecutive page number
\newcommand\pagenum{3}
% set title (only change date)
\title{Algorithmen \& Datenstrukturen - Aufgaben zum \\09. November 2015 (Blatt \leadingzero{\pagenum})}
% set authors
\author{Tobias Knöppler}

\renewcommand{\thesection}{\pagenum.\arabic{section}}
\renewcommand{\thesubsection}{~~\arabic{subsection}.}
\renewcommand{\thesubsubsection}{\roman{subsubsection})}
\titleformat{\subsubsection}[runin]{\normalfont\normalsize\bfseries}{\thesubsubsection}{1em}{}


\author{Tobias Knöppler (6523815), Nico Tress (6378086)}
\date{\gertoday}
\begin{document}
\maketitle

\section{}%section name}\label{label}
$ALGO1()$: Da die beiden inneren Schleifen für jeden Durchlauf der äußeren je n mal durchlaufen werden (also 2n zusammen) und die äußere Schleife n+1 mal durchlaufen wird. daraus folgt $ALGO1() \in O((n+1) * 2n) = O(2n^2 + 2n)$.\\
\\
$ALGO2()$: Da die äußere Schleife bis 2n läuft, i aber jeweils um 2 inkrementiert wird, wird die while-Schleife n mal durchlaufen. Die innere Schleife wird jeweils i mal ausgeführt, also insgesamt 
\begin{math}
  1 + 3 + 5 + ... + 2n-3 + 2n-1 = \sum\limits_{i=1}^n 2i - 1\\
  = 2n-1 + 2n-3 + ... + 4-1 + 2-1\\
  = 2n + 2n + ... + 4-1-3 + 2-1-1\\
  = \left\{\bfrac{
    (n-1) \cdot 2n \qquad\qquad \text{für gerade n}
  }{
    2n + (n-2) \cdot 2(n-1) ~ \text{für ungerade n}
  }\right.\\
  = \left\{\bfrac{
    2n^2-2n \qquad~~ \text{für gerade n}
  }{
    2n + 2n^2 - 8n + 4 ~ \text{für ungerade n}
  }\right.\\
  = \left\{\bfrac{
    2n^2-2n \qquad~~ \text{für gerade n}
  }{
    2n^2 - 6n + 4 ~ \text{für ungerade n}
  }\right.\\
\end{math}
 mal...\\
 Damit gilt $ALGO2() \in O(2n^2 - 6n + 4) \in O(2n^2)$.\\
 \\
$ALGO3()$: Die äußere Schleife wird genau $\sqrt{n}$ mal ausgeführt. Die innere Schleife wird jeweils sooft durchlaufen, wie n durch 2 geteilt werden kann, ohne dass eine Zahl kleiner oder gleich 1 resultiert. Dies entspricht dem Exponenten der zu n nächstgrößeren 2er-Potenz.
Daraus folgt für eine Zweierpotenz a mit $a > n, (a - n)$ so klein, wie möglich: $a = 2^x => ln(a) = x \cdot ln(2) => x = \frac{ln(a)}{ln(2)} = \frac{10}{7} \cdot ln(a) => ALGO3() \in O(\sqrt{n} \cdot \lceil \frac{10}{7} \cdot ln(n) \rceil)$

\section{}%section name}\label{label}

Dies ist eine Rekurrenzgleichung für den Algorithmus FUNC(A):\\
\begin{math}
T(x) := \left\{ \bfrac{1, \qquad~~~ wenn~n < 4}
          {5 + 3n + 2 \cdot T(\frac{n}{4}) ~~ sonst}\right.
\end{math}
\\\\
Sie ergibt sich daraus, dass bei einer Eingabegröße (Länge von A) $< 4$ lediglich 5 zurückgegeben wird; die Komplexität ist also 1.\\
Ansonsten gilt (für $n = \text{A.länge}$):
\begin{itemize}
    \item 5 'einfache' Anweisungen werden unabhängig von der Eingabe ausgeführt (Zeilen 4, 7, 10 und der Zuweisungsanteil in den Zeilen 8 und 9).
    \item Die beiden for-Schleifen werden beide je A.länge mal durchlaufen; die erste enthält eine; die zweite 2 Anweisungen, daraus ergibt sich eine Komplexität von $3 \cdot n$.
    \item FUNC ruft sich selbst zweimal mit je einem Viertel des Arrays auf, daraus ergibt sich eine Komplexität von $2 \cdot T(n/4)$
\end{itemize}

\section{}%section name}\label{label}
\subsection{}%subsection name}\label{label}

\begin{math}
    T_2(n) := \left\{\bfrac{
        c_1, \qquad\qquad~~ \text{für n=1}
    }{
        8 \cdot T_1\left(\frac{n}{2}\right) + d_1 \cdot n^3 \qquad \text{sonst}
    }\right.
\end{math}\\

daraus folgt nach dem Mastertheorem:\\

\begin{flalign*}
    a &= 8\\
    b &= 2\\
    f(n) &= d_1 \cdot n^3\\
    \rightarrow log_b(a) &= log_2(8) = 3
    \\
    \Rightarrow T_2(n) &\in \Theta(n^3 \cdot log_2(n))\text{, da } f(n) = d_1 \cdot n^3 \in \Theta(n^3)
\end{flalign*}

\subsection{}%subsection name}\label{label}
\begin{math}
    T_2(n) := \left\{\bfrac{
        c_2, \qquad\qquad~~ \text{für n=1}
    }{
        5 \cdot T_2\left(\frac{n}{4}\right) + d_2 \cdot n^2 \qquad \text{sonst}
    }\right.
\end{math}

\begin{flalign*}
    a &= 5\\
    b &= 4\\
    f(n) &= d_2 \cdot n^2\\
    \rightarrow log_b(a) &= log_4(5) = \frac{log(5)}{log(4} \approx 1,161
    \\
    \Rightarrow T_2(n) &\in \Theta(f(n)) \rightarrow T_2(n) \in \Theta(d_2 \cdot n^2)
    \text{, da } f(n) \in \Omega(n^{log_4(5) + \epsilon}) \text{ z.B. für } \epsilon = 1 \\
    \qquad \text{ wegen } 
        &\lim\limits_{n \rightarrow \infty}\left(\frac{
            d_2 \cdot n^2
        }{
            n^{1,161 + 1}
        }\right)\\
        = &\lim\limits_{n \rightarrow \infty}\left(\frac{
            d_2
        }{
            n^{0,161}
        }\right) = 0
\end{flalign*}

\subsection{}%subsection name}\label{label}
\begin{math}
    T_3(n) := \left\{\bfrac{
        c_3, \qquad\qquad~~ \text{für n=1}
    }{
        6 \cdot T_3\left(\frac{n}{3}\right) + d_3 \cdot n \cdot log(n) \qquad \text{sonst}
    }\right.
\end{math}

\begin{flalign*}
    a &= 6\\
    b &= 3\\
    f(n) &= d_3 \cdot n \cdot log(n)\\
    \rightarrow log_b(a) &= log_3(6) = \frac{log(6)}{log(3} \approx 1,631
    \\
    \Rightarrow T_3(n) &\in \Theta(f(n)) \rightarrow T_2(n) \in \Theta(d_2 \cdot n^2)
    \text{, da } f(n) \in \Omega(n^{log_4(5) + \epsilon}) \text{ z.B. für } \epsilon = 1 \\
    \qquad \text{ wegen } 
        &\lim\limits_{n \rightarrow \infty}\left(\frac{
            d_3 \cdot n \cdot log(n)
        }{
            n^{1,631}
        }\right)\\
        = &\lim\limits_{n \rightarrow \infty}\left(\frac{
            d_3 \cdot log(n)
        }{
            n^{0,631}
        }\right)\\
        \stackrel{H}{=} &\lim\limits_{n \rightarrow \infty}\left(
            \frac{d_3}{n}
        \cdot 
            \frac{n^{-0,369}}{0,631}
        \right)\\
        = &\lim\limits_{n \rightarrow \infty}\left(
            \frac{d_3 \cdot n^{0,369}}{0,631 \cdot n}
        \right)\\
        = &\lim\limits_{n \rightarrow \infty}\left(
            \frac{
                d_3 \cdot n^{0,369}
            }{
                0,631 \cdot n
            }
        \right)\\
        = &\lim\limits_{n \rightarrow \infty}\left(
            \frac{
                d_3
            }{
                0,631 \cdot n^{0,631}
            }
        \right)\\
        = &0
\end{flalign*}

\section{}%section name}\label{label}
\subsection{}%subsection name}\label{label}

\begin{lstlisting}
merge(A, B):
  O = empty list
  while A is not empty and B is not empty:
    if A[0] < B[0]:
      append A[0] to O
      remove A[0]
    else
      append B[0] to O
      remove B[0]
    if A is empty
      return concat(O, B)
\end{lstlisting}

\subsection{}%subsection name}\label{label}

\begin{lstlisting}
numberOfConflicts(A)
  pendMax = 0
  numConflicts = 0
  for i=1 to A.length
    
\end{lstlisting}
\end{document}
