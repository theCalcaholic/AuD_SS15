\documentclass[fleqn]{article}
\usepackage[utf8]{inputenc}
\usepackage{geometry}
\usepackage{fancyhdr}
\usepackage{amsmath,amsthm,amssymb}
\usepackage{graphicx}
\usepackage{hyperref}
\usepackage{lipsum}
\usepackage{ulem}
\usepackage{comment}
\usepackage{enumerate}
\usepackage{titlesec}

\date{\gertoday}

\lhead{\@author}
\chead{}
\rhead{\gertoday}
\lfoot{}
\cfoot{\thepage}
\rfoot{}
\setlength{\mathindent}{0pt}

\usepackage[x11names, rgb]{xcolor}
\usepackage[utf8]{inputenc}
\usepackage{tikz}
\usepackage{geometry}
\usepackage{fancyhdr}
\usepackage{amsmath,amsthm,amssymb}
\usepackage{graphicx}
\usepackage{hyperref}
\usepackage{lipsum}
\usepackage{ulem}
\usepackage{comment}
\usepackage{enumerate}
\usepackage{titlesec}
\usepackage{boolexpr,pdftexcmds,trace}
\makeatletter

\usetikzlibrary{snakes,arrows,shapes}
\newwrite\dotfile

\begingroup
  \catcode`\[ = 1\relax
  \catcode`\] = 2\relax
  \catcode`\{ = 12\relax
  \catcode`\} = 12 \relax
  \gdef\OpenBrace[{]
  \gdef\CloseBrace[}]
\endgroup

% custom commands
\newcommand{\leadingzero}[1]{\ifnum #1<10 0#1\else#1\fi}
\newcommand{\gerdate}[3]{\leadingzero{#1}.\leadingzero{#2}.\leadingzero{#3}}
\newcommand{\gertoday}{\gerdate{\the\day}{\the\month}{\the\year}}
\newcommand*{\bfrac}[2]{\genfrac{}{}{0pt}{}{#1}{#2}}
\newcommand{\R}{\mathbb{R}}
\newcommand{\N}{\mathbb{N}}
\newcommand{\Q}{\mathbb{Q}}
\newcommand{\dotarrow}[0]{}

\newenvironment{graphviz}[1]%
{%
\switch
\case{\pdf@strcmp{#1}{graph}}
    \renewcommand{\dotarrow}[0]{--}
\case{\pdf@strcmp{#1}{strict graph}}
    \renewcommand{\dotarrow}[0]{--}
\case{\pdf@strcmp{#1}{digraph}}
    \renewcommand{\dotarrow}[0]{->}
\case{\pdf@strcmp{#1}{strict digraph}}
    \renewcommand{\dotarrow}[0]{->}
\endswitch

\immediate\openout\dotfile=tmp.dot%
\newcommand{\node}[2]{%
\immediate\write\dotfile{##1 \dotarrow \OpenBrace##2\CloseBrace}%
}%
%
\immediate\write\dotfile{#1 \OpenBrace}
}%
{\immediate\write\dotfile{\CloseBrace}%
\immediate\closeout\dotfile%
\immediate\write18{dot2tex --figonly tmp.dot > tmp.tex}%
\input{tmp.tex}%
}

\setcounter{section}{0}
\setcounter{subsection}{0}
\pagestyle{fancy}

\lhead{}
\chead{}
\rhead{\gertoday}
\lfoot{}
\cfoot{\thepage}
\rfoot{}
\setlength{\mathindent}{0pt}

% document specific settings and formats
% set consecutive page number
\newcommand\pagenum{3}
% set title (only change date)
\title{Algorithmen \& Datenstrukturen - Aufgaben zum \\09. November 2015 (Blatt \leadingzero{\pagenum})}
% set authors
\author{Tobias Knöppler}

\renewcommand{\thesection}{\pagenum.\arabic{section}}
\renewcommand{\thesubsection}{\arabic{section}. \alph{subsection})}
\renewcommand{\thesubsubsection}{\roman{subsubsection})}
\titleformat{\subsubsection}[runin]{\normalfont\normalsize\bfseries}{\thesubsubsection}{1em}{}


\author{}
\date{\gertoday}
\begin{document}
\maketitle

\section{}%section name}\label{label}
$ALGO1()$: Da die beiden inneren Schleifen für jeden Durchlauf der äußeren je n mal durchlaufen werden (also 2n zusammen) und die äußere Schleife n+1 mal durchlaufen wird. daraus folgt $ALGO1() \in O((n+1) * 2n) = O(2n^2 + 2n)$.\\
\\
$ALGO2()$: Da die äußere Schleife bis 2n läuft, i aber jeweils um 2 inkrementiert wird, wird die while-Schleife n mal durchlaufen. Die innere Schleife wird jeweils i mal ausgeführt, also insgesamt \[\sum\limits_{i=1}^n 2i = 1 + 3 + 5 + ... + 2n-3 + 2n-1 = 2n + 2n + ... + 3 - 3 + 1 - 1 = (2n)^{n-1}\] mal...\\
\\
$ALGO3()$: Die äußere Schleife wird genau $\sqrt{n}$ mal ausgeführt. Die innere Schleife wird jeweils sooft durchlaufen, wie n durch 2 geteilt werden kann, ohne dass eine Zahl kleiner oder gleich 1 resultiert. Dies entspricht dem Exponenten der zu n nächstgrößeren 2er-Potenz.
Daraus folgt für eine Zweierpotenz a mit $a > n, (a - n)$ so klein, wie möglich: $a = 2^x => ln(a) = x \cdot ln(2) => x = \frac{ln(a)}{ln(2)} = \frac{10}{7} \cdot ln(a) => ALGO3() \in O(\sqrt{n} \cdot \lceil \frac{10}{7} \cdot ln(n) \rceil)$

\section{}%section name}\label{label}

\begin{flalign}
T(x) := \left\{ 1 &, wenn n < 4 \\
          3 + 3n + 2 \cdot T(\frac{n}{4})\right.
\end{flalign}

\end{document}
