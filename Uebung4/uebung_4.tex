\documentclass[fleqn]{article}
\usepackage[utf8]{inputenc}
\usepackage{geometry}
\usepackage{fancyhdr}
\usepackage{amsmath,amsthm,amssymb}
\usepackage{graphicx}
\usepackage{hyperref}
\usepackage{lipsum}
\usepackage{ulem}
\usepackage{comment}
\usepackage{enumerate}
\usepackage{titlesec}
\usepackage{listings}

\lstset{
  basicstyle=\scriptsize\ttfamily,
  keywordstyle=\bfseries\ttfamily\color{orange},
  stringstyle=\color{green}\ttfamily,
  commentstyle=\color{middlegray}\ttfamily,
  emph={square}, 
  emphstyle=\color{blue}\texttt,
  emph={[2]root,base},
  emphstyle={[2]\color{yac}\texttt},
  showstringspaces=false,
  flexiblecolumns=false,
  tabsize=2,
  numbers=left,
  numberstyle=\tiny,
  numberblanklines=false,
  stepnumber=1,
  numbersep=6pt,
  xleftmargin=15pt,
  literate=%
    {Ö}{{\"O}}1
    {Ä}{{\"A}}1
    {Ü}{{\"U}}1
    {ß}{{\ss}}1
    {ü}{{\"u}}1
    {ä}{{\"a}}1
    {ö}{{\"o}}1
    {~}{{\textasciitilde}}1
}

\date{\gertoday}

\lhead{\@author}
\chead{}
\rhead{\gertoday}
\lfoot{}
\cfoot{\thepage}
\rfoot{}
\setlength{\mathindent}{0pt}

\usepackage[x11names, rgb]{xcolor}
\usepackage[utf8]{inputenc}
\usepackage{tikz}
\usepackage{geometry}
\usepackage{fancyhdr}
\usepackage{amsmath,amsthm,amssymb}
\usepackage{graphicx}
\usepackage{hyperref}
\usepackage{lipsum}
\usepackage{ulem}
\usepackage{comment}
\usepackage{enumerate}
\usepackage{titlesec}
\usepackage{boolexpr,pdftexcmds,trace}
\makeatletter

\usetikzlibrary{snakes,arrows,shapes}
\newwrite\dotfile

\begingroup
  \catcode`\[ = 1\relax
  \catcode`\] = 2\relax
  \catcode`\{ = 12\relax
  \catcode`\} = 12 \relax
  \gdef\OpenBrace[{]
  \gdef\CloseBrace[}]
\endgroup

% custom commands
\newcommand{\leadingzero}[1]{\ifnum #1<10 0#1\else#1\fi}
\newcommand{\gerdate}[3]{\leadingzero{#1}.\leadingzero{#2}.\leadingzero{#3}}
\newcommand{\gertoday}{\gerdate{\the\day}{\the\month}{\the\year}}
\newcommand*{\bfrac}[2]{\genfrac{}{}{0pt}{}{#1}{#2}}
\newcommand{\R}{\mathbb{R}}
\newcommand{\N}{\mathbb{N}}
\newcommand{\Q}{\mathbb{Q}}
\newcommand{\dotarrow}[0]{}

\newenvironment{graphviz}[1]%
{%
\switch
\case{\pdf@strcmp{#1}{graph}}
    \renewcommand{\dotarrow}[0]{--}
\case{\pdf@strcmp{#1}{strict graph}}
    \renewcommand{\dotarrow}[0]{--}
\case{\pdf@strcmp{#1}{digraph}}
    \renewcommand{\dotarrow}[0]{->}
\case{\pdf@strcmp{#1}{strict digraph}}
    \renewcommand{\dotarrow}[0]{->}
\endswitch

\immediate\openout\dotfile=tmp.dot%
\newcommand{\node}[2]{%
\immediate\write\dotfile{##1 \dotarrow \OpenBrace##2\CloseBrace}%
}%
%
\immediate\write\dotfile{#1 \OpenBrace}
}%
{\immediate\write\dotfile{\CloseBrace}%
\immediate\closeout\dotfile%
\immediate\write18{dot2tex --figonly tmp.dot > tmp.tex}%
\input{tmp.tex}%
}

\setcounter{section}{0}
\setcounter{subsection}{0}
\pagestyle{fancy}

\lhead{}
\chead{}
\rhead{\gertoday}
\lfoot{}
\cfoot{\thepage}
\rfoot{}
\setlength{\mathindent}{0pt}

% document specific settings and formats
% set consecutive page number
\newcommand\pagenum{4}
% set title (only change date)
\title{Algorithmen \& Datenstrukturen - Aufgaben zum \\23. November 2015 (Blatt \leadingzero{\pagenum})}
% set authors
\author{Tobias Knöppler}

\renewcommand{\thesection}{\pagenum.\arabic{section}}
\renewcommand{\thesubsection}{~~\arabic{subsection}.}
\renewcommand{\thesubsubsection}{\roman{subsubsection})}
\titleformat{\subsubsection}[runin]{\normalfont\normalsize\bfseries}{\thesubsubsection}{1em}{}


\author{Tobias Knöppler (6523815), Nico Tress (6378086)}
\date{\gertoday}
\begin{document}
\maketitle
\section{}%section name}\label{label}
\subsection{}%subsection name}\label{label}

Es existiert genau dann eine Partitionierung von U, die die geforderten Eigenschaften hat, wenn $\forall u \in U (s(u) \leq K)$ gilt. Deshalb genügt der folgende Algorithmus für die Entscheidung, ob es eine Partitionierung gibt, sodass für jede Partition $\sum\limits_{u \in U_i} s(u) \leq K$ erfüllt ist:

Pseudocode:
\begin{lstlisting}
ExistsPartition(U, K):
  for each u in U
    if s(u) > K
      return false
  return true
\end{lstlisting}

Es lässt sich leicht sehen, dass dieser Algorithmus in NP ist; seine Laufzeit ist sogar in O(|U|), da er nur eine Schleife über alle Element in U enthält.

\subsection{}%subsection name}\label{label}
Für diese Entscheidung genügt es, diejenigen Teilworte der Worte aus S zu betrachten, deren Länge exakt n ist, da alle längeren Teilworte mindestens eines der Teilworte der Länge n selbst als Teilwort enthielten.\\
Die Anzahl der Teilworte der Länge n eines Wortes $x \in S$ ist nun gegeben durch $|x| - n + 1$. Dies lässt sich leicht einsehen, wenn man überlegt, wie oft man eine "Schablone" der Länge n auf ein Wort x legen und diese verschieben könnte, ohne dass sie darüber hinaus ragt. für eine Schablone der Länge 1 gäbe es dafür exakt $|x|$ Möglichkeiten ($|x| - 1 + 1$); für eine um $y$ längere Schablone entsprechend (y-1) Optionen mehr.\\
Mit dieser Vorüberlegung lässt sich der folgende Algorithmus formulieren:\\
\newpage
Geg. Menge von Worten S und Länge n
\begin{lstlisting}
ExistiertTeilwort(S, n)
  # iteriere über alle Worte in S
  for each x1 in S
    # Für jedes Teilwort w der Länge n von x1...
    for i=0 to (|x1| - n + 1)
      w = x1[i to (i + n-1)
      # ...iteriere über jedes Wort (x2) in S und...
      for each x2 in S
        # ...vergleiche w mit jedem Teilwort von x2.
        found_w = false
        for j=0 to (|x2| - n +1)
          # Falls x2 w enthält setze found_w = true
          if x2[i to (i + n - 1)] = w
            found_w = true
        # Falls ein Wort w nicht enthielt, setze w_in_all_x2 auf false
        if not found_w
          w_in_all_x2 = false
    # Falls eines der Teilworte in allen Worten aus S enthalten war, gebe true zurück.
    if w_in_all_x2
      return true
\end{lstlisting}

Liese man diesen Algorithmus nicht-deterministisch ausführen, indem jede Iteration der ersten for-Schleife gleichzeitig ausgeführt wird, und true zurückgegeben wird, sobald eine For-Schleife true zurückgibt (bzw. false, wenn alle false zurückgeben), so lässt sich leicht sehen, dass der Algorithmus in NP ist (auch wenn es wesentlich effizientere Algorithmen gäbe).

\section{}%section name}\label{label}
\subsection{}%subsection name}\label{label}

\[
  L_1, L_2 \in P \Rightarrow L_1 \in O(n^a) \wedge L_2 \in O(n^b) \Rightarrow L_1 \cdot L_2 \in O(n^a + n^b) \in O(n^{max (a, b) + 1}) \Rightarrow L_1 \cdot L_2 \in P ~\square
\]\\
\\
Seien $L_1, L_2, L_3, L_4$ Sprachen, sodass $L_1, L_2 \in P \wedge L_3, L_4 \in NP$.\\
Dann gibt es zwei DTMs S,T, und zwei NTMs U,V, sodass S alle Worte von $L_1$, T alle Worte von $L_2$, U alle Worte von $L_3$ und V alle Worte von $L_4$ akzeptiert.\\
Für jede der TMs existiert zudem ein Polynom, dass dessen Laufzeit beschränkt.\\
\\
Nun lässt sich leicht durch aneinanderfügen der Turingmaschinen S und T - indem von jedem Endzustand von S eine Kante zu dem Startzustand von T hinzugefügt wird und die jeweiligen Ein- und Ausgaben auf je einem eigenen Band liegen - eine weitere DTM erstellen, die S und T nacheinander lößt.\\
Ähnlich lässt sich für U und V - durch Verbinden jedes Endknotens von U mit dem Startknoten von V - eine NTM erzeugen, die $U \cdot V$ lößt.\\
Damit ist $\forall S, T (S \cdot T \in P)$ und $\forall U, V (U \cdot V \in NP)$ bewiesen. $\square$

\newpage
\subsection{}%subsection name}\label{label}
Geg. Aussagenlogische Formel F (n sei die Anzahl der Literale
\begin{lstlisting}
getAssignment(F)
  i = 1
  while count(c of size i in F) != 0
    # Falls eine Klausel leer ist (d.h. sie enthält keine nicht zu false evaluierten Liteale),
    # gibt es keine gültige Belegung
    for each clause c in F # Schleifenkomplexität: |F|
      if c is empty
        return false
    # Iteriere über alle Klauseln der Länge i, wobei i der Länge der kleinsten Klauseln,
    # welche noch negierte Literale enthalten können, entspricht.
    for each clause c of length i in F # Schleifenkomplexität: |F|
      # Sobald ein negiertes Literal gefunden wird, wird es mit true (das nicht-negierte
      # also mit false) belegt.
      for each literal l in c # Schleifenkomplexität: |c|
        if l is negated
          for each clause d in F # Schleifenkomplexität: |F|
            # Dadurch kann es überall dort, wo es nicht-negiert vorkommt
            # (entspricht "or false") gelöscht werden;
            if d contains negated l # Komplexität: |d|
              remove d
            # überall, wo es negiert vorkommt
            # (entspricht "or true"), kann die Klausel gelöscht werden
            else if d contains non-negated l # Komplexität: |d|
              remove l from d
          # Danach wird die while-Schleife "neugestartet" (mit i=1)
          i = 1
          continue while;
    
\end{lstlisting}

n sei die Anzahl der Literale
Die Laufzeit eines while-Schleifen-Durchlaufs ist im Worst Case in $O(n^2 + |F|) \in O(n^2)$, da die zweite For-Schleife (und die verschachtelten Schleifen) alle Literale mit allen Literalen vergleichen (im Worst Case) und die erste For-Schleife einmal über alle Klauseln iteriert.\\
Die while-Schleife selbst, hat die Komplexität $n + m$, wobei n der Anzahl Literale und m der maximalen Klausellänge entspricht.\\
Im Worst Case ist also offensichtlich n = m, woraus sich eine Gesamtkomplexität in $O(2n \cdot n^2 + |F|) \in O(n^3)$ ergibt.\\
Damit ist der Algorithmus in P. $\square$\\
\\
Unter der Annahme $P = NP$ lässt sich sogar ein wesentlich einfacher Algorithmus finden: Eine NTM kann nicht-deterministisch jede mögliche Belegung durchprobieren und wird in n Schritten terminieren und die richtige Belegung zurückgeben, wenn diese existiert. (*)

\section{}%section name}\label{label}
\subsection{}%subsection name}\label{label}
Sat-2 lässt sich folgendermaßen auf SAT reduzieren.\\
Man finde eine gültige Belegung für eine Formel F in Sat-2.\\
Nun sei $G = F \wedge \neg(a, b, c ...) \wedge (d, e, f ...)$, wobei $(a, b, c ...) $ den Literalen entsprechen, die bei der ersten Belegung mit true belegt wurden und $(d,e, f)$ den Literalen, welche bei der ersten Belegung mit false belegt wurden.\\
Bringt man nun G in die KNF und löst die daraus resultierende Formel (mittels SAT), so hat man (falls möglich) eine zweite gültige Belegung erhalten. Falls dies möglich war, ist F in SAT-2, falls nicht, ist F nicht in SAT-2. Damit ist SAT-2 auf SAT reduziert, woraus $SAT-2 \in NP-C$ folgt. $\square$

\end{document}
