\documentclass[fleqn]{article}
\usepackage[utf8]{inputenc}
\usepackage{geometry}
\usepackage{fancyhdr}
\usepackage{amsmath,amsthm,amssymb}
\usepackage{graphicx}
\usepackage{hyperref}
\usepackage{lipsum}
\usepackage{ulem}
\usepackage{comment}
\usepackage{enumerate}
\usepackage{titlesec}
\usepackage{listings}
\usepackage{parallel}

\lstset{
  basicstyle=\scriptsize\ttfamily,
  keywordstyle=\bfseries\ttfamily\color{orange},
  stringstyle=\color{green}\ttfamily,
  commentstyle=\color{middlegray}\ttfamily,
  emph={square}, 
  emphstyle=\color{blue}\texttt,
  emph={[2]root,base},
  emphstyle={[2]\color{yac}\texttt},
  showstringspaces=false,
  flexiblecolumns=false,
  tabsize=2,
  numbers=left,
  numberstyle=\tiny,
  numberblanklines=false,
  stepnumber=1,
  numbersep=6pt,
  xleftmargin=15pt,
  literate=%
    {Ö}{{\"O}}1
    {Ä}{{\"A}}1
    {Ü}{{\"U}}1
    {ß}{{\ss}}1
    {ü}{{\"u}}1
    {ä}{{\"a}}1
    {ö}{{\"o}}1
    {~}{{\textasciitilde}}1
}

\date{\gertoday}

\lhead{\@author}
\chead{}
\rhead{\gertoday}
\lfoot{}
\cfoot{\thepage}
\rfoot{}
\setlength{\mathindent}{0pt}

\usepackage[x11names, rgb]{xcolor}
\usepackage[utf8]{inputenc}
\usepackage{tikz}
\usepackage{geometry}
\usepackage{fancyhdr}
\usepackage{amsmath,amsthm,amssymb}
\usepackage{graphicx}
\usepackage{hyperref}
\usepackage{lipsum}
\usepackage{ulem}
\usepackage{comment}
\usepackage{enumerate}
\usepackage{titlesec}
\usepackage{boolexpr,pdftexcmds,trace}
\makeatletter

\tikzset{>=latex} 
\usetikzlibrary{snakes,arrows,shapes,positioning}
\newwrite\dotfile

\begingroup
  \catcode`\[ = 1\relax
  \catcode`\] = 2\relax
  \catcode`\{ = 12\relax
  \catcode`\} = 12 \relax
  \gdef\OpenBrace[{]
  \gdef\CloseBrace[}]
\endgroup

% custom commands
\newcommand{\leadingzero}[1]{\ifnum #1<10 0#1\else#1\fi}
\newcommand{\gerdate}[3]{\leadingzero{#1}.\leadingzero{#2}.\leadingzero{#3}}
\newcommand{\gertoday}{\gerdate{\the\day}{\the\month}{\the\year}}
\newcommand*{\bfrac}[2]{\genfrac{}{}{0pt}{}{#1}{#2}}
\newcommand{\R}{\mathbb{R}}
\newcommand{\N}{\mathbb{N}}
\newcommand{\Q}{\mathbb{Q}}
\newcommand{\dotarrow}[0]{}

\newenvironment{graphviz}[1]%
{%
\switch
\case{\pdf@strcmp{#1}{graph}}
    \renewcommand{\dotarrow}[0]{--}
\case{\pdf@strcmp{#1}{strict graph}}
    \renewcommand{\dotarrow}[0]{--}
\case{\pdf@strcmp{#1}{digraph}}
    \renewcommand{\dotarrow}[0]{->}
\case{\pdf@strcmp{#1}{strict digraph}}
    \renewcommand{\dotarrow}[0]{->}
\endswitch

\immediate\openout\dotfile=tmp.dot%
\newcommand{\node}[2]{%
\immediate\write\dotfile{##1 \dotarrow \OpenBrace##2\CloseBrace}%
}%
%
\immediate\write\dotfile{#1 \OpenBrace}
}%
{\immediate\write\dotfile{\CloseBrace}%
\immediate\closeout\dotfile%
\immediate\write18{dot2tex --figonly tmp.dot > tmp.tex}%
\input{tmp.tex}%
}

\setcounter{section}{0}
\setcounter{subsection}{0}
\pagestyle{fancy}

\lhead{}
\chead{}
\rhead{\gertoday}
\lfoot{}
\cfoot{\thepage}
\rfoot{}
\setlength{\mathindent}{0pt}

% document specific settings and formats
% set consecutive page number
\newcommand\pagenum{5}
% set title (only change date)
\title{Algorithmen \& Datenstrukturen - Aufgaben zum \\7. Dezember 2015 (Blatt \leadingzero{\pagenum})}
% set authors
\author{Tobias Knöppler}

\renewcommand{\thesection}{\pagenum.\arabic{section}}
\renewcommand{\thesubsection}{~~\arabic{subsection}.}
\renewcommand{\thesubsubsection}{\roman{subsubsection})}
\titleformat{\subsubsection}[runin]{\normalfont\normalsize\bfseries}{\thesubsubsection}{1em}{}


\author{Tobias Knöppler (6523815), Nico Tress (6378086)}
\date{\gertoday}
\begin{document}
\maketitle

\section{}
\subsection{}%subsection name}\label{label}

\begin{enumerate}[i)]
\item 
\begin{tikzpicture}[<-]
  \tikzstyle{elem} = [rectangle, draw]

  \node[elem] (0) {5};
\end{tikzpicture}

\item 
\begin{tikzpicture}[<-]
  \tikzstyle{elem} = [rectangle, draw]

  \node[elem] (n0) {5};
  \node[elem] (n1)
    [below right=5mm and 5mm of n0.south] 
    {8} 
    edge (n0);
\end{tikzpicture}

\item
\begin{tikzpicture}[<-]
  \tikzstyle{elem} = [rectangle, draw]

  \node[elem] (n0) {5};
  \node[elem] (n1)
    [below right=5mm and 5mm of n0.south] 
    {8} 
    edge (n0);
  \node[elem] (n2)
    [below right=5mm and 5mm of n1.south]
    {9}
    edge (n1);
\end{tikzpicture}

\item
\begin{tikzpicture}[<-]
  \tikzstyle{elem} = [rectangle, draw]

  \node[elem] (n0) {5};
  \node[elem] (n1)
    [below right=5mm and 5mm of n0.south] 
    {8} 
    edge (n0);
  \node[elem] (n2)
    [below right=5mm and 5mm of n1.south]
    {9}
    edge (n1);
  \node[elem] (n3)
    [below left=5mm and 5mm of n1.south]
    {6}
    edge (n1);
\end{tikzpicture}

\item
\begin{tikzpicture}[<-]
  \tikzstyle{elem} = [rectangle, draw]

  \node[elem] (n0) {5};
  \node[elem] (n1)
    [below right=5mm and 5mm of n0.south] 
    {8} 
    edge (n0);
  \node[elem] (n2)
    [below right=5mm and 5mm of n1.south]
    {9}
    edge (n1);
  \node[elem] (n3)
    [below left=5mm and 5mm of n1.south]
    {6}
    edge (n1);
  \node[elem] (n4)
    [below right=5mm and 5mm of n3.south]
    {7}
    edge (n3);
\end{tikzpicture}
\end{enumerate}

\subsection{}%subsection name}\label{label}
\begin{enumerate}[i)]
\item
\begin{tikzpicture}[<-]
  \tikzstyle{elem} = [rectangle, draw]
  \node[elem] (n0)
    {20};
  \node[elem] (n1)
    [below left=5mm and 15mm of n0.south]
    {10}
    edge (n0);
  \node[elem] (n2)
    [below right=5mm and 15mm of n0.south]
    {25}
    edge (n0);
  \node[elem] (n3)
    [below left=5mm and 5mm of n1.south]
    {7}
    edge (n1);
  %\node[elem] (n4)
  %  [below right=5mm and 5mm of n1.south]
  %  {14}
  %  edge (n1);
  \node[elem] (n5)
    [below left=5mm and 5mm of n2.south]
    {22}
    edge (n2);
  \node[elem] (n6)
    [below right=5mm and 5mm of n2.south]
    {27}
    edge (n2);
  \node[elem] (n7)
    [below left=5mm and 5mm of n5.south]
    {21}
    edge (n5);
  \node[elem] (n8)
    [below right=5mm and 5mm of n5.south]
    {24}
    edge (n5);
  \node[elem] (n9)
    [below right=5mm and 5mm of n6.south]
    {29}
    edge (n6);
  \node[elem] (n10)
    [below left=5mm and 5mm of n9.south]
    {28}
    edge (n9);
  \node[elem] (n11)
    [below right=5mm and 5mm of n9.south]
    {31}
    edge (n9);
\end{tikzpicture}

\item
\begin{tikzpicture}[<-]
  \tikzstyle{elem} = [rectangle, draw]
  \node[elem] (n0)
    {20};
  \node[elem] (n1)
    [below left=5mm and 15mm of n0.south]
    {10}
    edge (n0);
  \node[elem] (n2)
    [below right=5mm and 15mm of n0.south]
    {25}
    edge (n0);
  \node[elem] (n3)
    [below left=5mm and 5mm of n1.south]
    {7}
    edge (n1);
  %\node[elem] (n4)
  %  [below right=5mm and 5mm of n1.south]
  %  {14}
  %  edge (n1);
  \node[elem] (n5)
    [below left=5mm and 5mm of n2.south]
    {22}
    edge (n2);
  %\node[elem] (n6)
  %  [below right=5mm and 5mm of n2.south]
  %  {27}
  %  edge (n2);
  \node[elem] (n7)
    [below left=5mm and 1mm of n5.south]
    {21}
    edge (n5);
  \node[elem] (n8)
    [below right=5mm and 1mm of n5.south]
    {24}
    edge (n5);
  \node[elem] (n9)
    [below right=5mm and 5mm of n2.south]
    {29}
    edge (n2);
  \node[elem] (n10)
    [below left=5mm and 1mm of n9.south]
    {28}
    edge (n9);
  \node[elem] (n11)
    [below right=5mm and 1mm of n9.south]
    {31}
    edge (n9);
\end{tikzpicture}

\item
\begin{tikzpicture}[<-]
  \tikzstyle{elem} = [rectangle, draw]
  \node[elem] (n0)
    {20};
  \node[elem] (n1)
    [below left=5mm and 15mm of n0.south]
    {10}
    edge (n0);
  \node[elem] (n2)
    [below right=5mm and 15mm of n0.south]
    {24}
    edge (n0);
  \node[elem] (n3)
    [below left=5mm and 5mm of n1.south]
    {7}
    edge (n1);
  %\node[elem] (n4)
  %  [below right=5mm and 5mm of n1.south]
  %  {14}
  %  edge (n1);
  \node[elem] (n5)
    [below left=5mm and 5mm of n2.south]
    {22}
    edge (n2);
  %\node[elem] (n6)
  %  [below right=5mm and 5mm of n2.south]
  %  {27}
  %  edge (n2);
  \node[elem] (n7)
    [below left=5mm and 1mm of n5.south]
    {21}
    edge (n5);
  %\node[elem] (n8)
  %  [below right=5mm and 1mm of n5.south]
  %  {24}
  %  edge (n5);
  \node[elem] (n9)
    [below right=5mm and 5mm of n2.south]
    {29}
    edge (n2);
  \node[elem] (n10)
    [below left=5mm and 1mm of n9.south]
    {28}
    edge (n9);
  \node[elem] (n11)
    [below right=5mm and 1mm of n9.south]
    {31}
    edge (n9);
\end{tikzpicture}
\end{enumerate}

\section{}%section name}\label{label}
\subsection{}%subsection name}\label{label}
\begin{lstlisting}
InorderTreeWalk(x)
  for each sibling n of x
    n.color = white
  
  while x.color == white or right[x].color == white
    if left[x] != nil and left[x].color == white
      x = left[x]
    else
      if x.color == white
        print key[x]
        x.color = black
      if right[x] != nil and right[x].color == white
        x = right[x]
\end{lstlisting}
\subsection{}%subsection name}\label{label}

\begin{lstlisting}
TreeMinimum(x)
  if links[x] == nil
    return x
  else
    return TreeMinimum(left[x])
\end{lstlisting}

\section{}%section name}\label{label}
\begin{enumerate}[I]
  \item Ein Pfad hat maximal die Länge $n-1$, wobei n die Anzahl Knoten ist, die er durchläuft.
  \item Wenn es in einem Graphen G zwei Knoten s und t gibt, die durch einen oder mehrere  Pfade verbunden sind, so gibt es genau dann einen Knoten v, bei dessen Entfernung alle Pfade von s nach t unterbrochen würden, wenn all diese Pfade v enthalten.
  \item Angenommen, G sei ein Graph mit n Knoten. s und t seien 2 Knoten in G, zweischen denen es 2 Pfade p, q mit $|p|,|q| > \frac{n}{2}$ gibt, die außer s und t keine Knoten gemeinsam haben. Dann enthält der kleinere Pfad von p und q außer s und t maximal $\frac{n}{2}$, also $\frac{n - 2}{2} + 2$ Knoten (nämlich dann, wenn p und q jeweils genau die Hälfte der übrigen Knoten enthalten) und hat wegen $I$ damit eine maximale Länge von $\frac{n - 2}{2} + 1 = \frac{n}{2}$.\\
  Damit ist die Annahme, dass $|p|,|q| > \frac{n}{2}$ zum Widerspruch geführt und bewiesen, dass zwei Pfade, die s und t verbinden und länger als $\frac{n}{2}$ sind, mindestens einen Knoten gemeinsam haben müssen.
  \item Aus $II$ und $III$ folgt, dass es in einem Graphen mit n Knoten, der zwei Knoten s und t enthält, deren Distanz $> \frac{n}{2}$ ist, einen Knoten v geben muss, der allen Pfaden zweischen s und t gemein ist, wodurch durch Löschen von v alle Pfade zwischen s und t zerstört werden. $\square$
\end{enumerate}

\section{}%section name}\label{label}
\begin{lstlisting}
FindV(s, t, V)
  for each n in V
    n.counter = 0
    n.color = white
  s.color = black
  
  for each n in adjacents(s)
    CheckPath(n, t)
    
  v = random node in V
  for each n in V
    if v.counter < n.counter
      v = n
  
  return v
  
  
CheckPath(n, t)
  
  if n == t
    return 1

  n.color = grey
  for each m in adjacents(n)
    if m.color == white 
      paths = CheckPath(m, t)
      n.counter = n.counter + paths
    else if m.color == black and m.counter > 0
      n.counter = n.counter + m.counter
  m.color = black
  
  return n.counter
  
\end{lstlisting}

FindV durchläuft die Knoten in G in einer Tiefensuche, ausgehend von den angrenzenden Knoten von s. Dabei wird für jeden Knoten überprüft, ob von diesem ein oder mehrere Pfade zu t führen; falls dies der Fall ist, so wird der Counter des aktuellen Knotens um 1 für jeden gefundenen Pfad zu t erhöht (Z. 26-27). \\
Stößt der Algorithmus dabei auf einen bereits markierten Knoten, so addiert er die Anzahl der Pfade zu t, von diesem Knoten zu dem eigenen Counter (Z. 28-29).\\
Am Ende erhalten so alle von s erreichbaren Knoten einen counter, der den Pfaden von s zu t über diesen Knoten entspricht. Nun genügt es den Knoten mit dem höchsten counter zu finden und zurückzugeben (Z. 15) - haben mehrere Knoten den höchsten counter, so ist ein beliebiger davon gültig.
\end{document}
