\documentclass[fleqn]{article}
\usepackage[utf8]{inputenc}
\usepackage{geometry}
\usepackage{fancyhdr}
\usepackage{amsmath,amsthm,amssymb}
\usepackage{graphicx}
\usepackage{hyperref}
\usepackage{lipsum}
\usepackage{ulem}
\usepackage{comment}
\usepackage{enumerate}
\usepackage{titlesec}

\date{\gertoday}

\lhead{\@author}
\chead{}
\rhead{\gertoday}
\lfoot{}
\cfoot{\thepage}
\rfoot{}
\setlength{\mathindent}{0pt}

\usepackage[x11names, rgb]{xcolor}
\usepackage[utf8]{inputenc}
\usepackage{tikz}
\usepackage{geometry}
\usepackage{fancyhdr}
\usepackage{amsmath,amsthm,amssymb}
\usepackage{graphicx}
\usepackage{hyperref}
\usepackage{lipsum}
\usepackage{ulem}
\usepackage{comment}
\usepackage{enumerate}
\usepackage{titlesec}
\usepackage{boolexpr,pdftexcmds,trace}
\makeatletter

\usetikzlibrary{snakes,arrows,shapes}
\newwrite\dotfile

\begingroup
  \catcode`\[ = 1\relax
  \catcode`\] = 2\relax
  \catcode`\{ = 12\relax
  \catcode`\} = 12 \relax
  \gdef\OpenBrace[{]
  \gdef\CloseBrace[}]
\endgroup

% custom commands
\newcommand{\leadingzero}[1]{\ifnum #1<10 0#1\else#1\fi}
\newcommand{\gerdate}[3]{\leadingzero{#1}.\leadingzero{#2}.\leadingzero{#3}}
\newcommand{\gertoday}{\gerdate{\the\day}{\the\month}{\the\year}}
\newcommand*{\bfrac}[2]{\genfrac{}{}{0pt}{}{#1}{#2}}
\newcommand{\R}{\mathbb{R}}
\newcommand{\N}{\mathbb{N}}
\newcommand{\Q}{\mathbb{Q}}
\newcommand{\dotarrow}[0]{}

\newenvironment{graphviz}[1]%
{%
\switch
\case{\pdf@strcmp{#1}{graph}}
    \renewcommand{\dotarrow}[0]{--}
\case{\pdf@strcmp{#1}{strict graph}}
    \renewcommand{\dotarrow}[0]{--}
\case{\pdf@strcmp{#1}{digraph}}
    \renewcommand{\dotarrow}[0]{->}
\case{\pdf@strcmp{#1}{strict digraph}}
    \renewcommand{\dotarrow}[0]{->}
\endswitch

\immediate\openout\dotfile=tmp.dot%
\newcommand{\node}[2]{%
\immediate\write\dotfile{##1 \dotarrow \OpenBrace##2\CloseBrace}%
}%
%
\immediate\write\dotfile{#1 \OpenBrace}
}%
{\immediate\write\dotfile{\CloseBrace}%
\immediate\closeout\dotfile%
\immediate\write18{dot2tex --figonly tmp.dot > tmp.tex}%
\input{tmp.tex}%
}

\setcounter{section}{0}
\setcounter{subsection}{0}
\pagestyle{fancy}

\lhead{}
\chead{}
\rhead{\gertoday}
\lfoot{}
\cfoot{\thepage}
\rfoot{}
\setlength{\mathindent}{0pt}

% document specific settings and formats
% set consecutive page number
\newcommand\pagenum{2}
% set title (only change date)
\title{Algorithmen \& Datenstrukturen - Aufgaben zum \\26. Oktober 2015 (Blatt \leadingzero{\pagenum})}
% set authors
\author{Tobias Knöppler}

\renewcommand{\thesection}{\pagenum.\arabic{section}}
\renewcommand{\thesubsection}{~~\arabic{subsection}.}
\renewcommand{\thesubsubsection}{\alph{subsubsection})}
\titleformat{\subsubsection}[runin]{\normalfont\normalsize\bfseries}{\thesubsubsection}{1em}{}


\author{Tobias Knöppler (6523815), Nico Tress (6378086)}
\date{\gertoday}
\begin{document}
\maketitle

\section{}
\subsection{}%subsection name}\label{label}
\begin{math}
        \lim\limits_{n \to \infty}\left(
            \frac{3n^3 - 6n + 20}{n^3}
        \right)
    ~=~ \lim\limits_{n \to \infty}\left(
            \frac{
                3 - \frac{6}{n^2} + \frac{20}{n^3}
            }{
                1
            }
        \right)
    ~=~ 3 \\\\
    \Rightarrow \text{Es existiert ein endlicher Limes $> 0$ für die Division der beiden Ausdrücke;} \\
    \text{d.h. $3n^3 - 6n + 20 \in O(n^3)$.}
\end{math}

\subsection{}%subsection name}\label{label}
\begin{math}
        \lim\limits_{n \to \infty}\left(
            \frac{
                n^2 \cdot log ~ n
            }{
                n^3
            }
        \right)
    ~=~ \lim\limits_{n \to \infty}\left(
            \frac{
                1 \cdot \frac{log ~ n}{n^2}
            }{
                n
            }
        \right)
    ~=~ \lim\limits_{n \to \infty}\left(
            \frac{log ~ n}{n^3}
        \right)
    ~\stackrel{H}{=}~ \lim\limits_{n \to \infty}\left(
            \frac{
                \frac{1}{n \cdot ln 10}
            }{
                3n^2
            }
        \right)
    ~=~ \lim\limits_{n \to \infty}\left(
            \frac{
                1
            }{
                3n^3 \cdot ln 10
            }\right)
    ~=~ \infty \\\\
    \Rightarrow \text{Es existiert kein endlicher Limes; d.h. $n^2 \cdot log ~ n \notin O(n^3) \rightarrow n^2 \cdot log ~ n \notin O(n^3) \cap \Omega(n^2)$.}
\end{math}

\section{}%section name}\label{label}
\subsection{}%subsection name}\label{label}
$log ~ n$ $<$ $ln ~ n$ $<$ 4 $<$ 1000 $<$ $\sqrt{n}^3 ~=~ n^\frac{1}{3}$ $<$ $n^{\frac{1}{2}}$ $<$ $n$ $<$ $n^2$ $<$ $2^n$ \\
\\
Anmerkung: $<$ drückt hier lediglich eine Ordnung aus, nicht "kleiner/gleich".\\
\\
Begründung:\\
$log ~ n \in O(ln ~ n)$, da $\lim\limits_{n \to \infty}
        \left(
            \frac{(log ~ n)'}{(ln ~ )'}
        \right)
    ~=~ \lim\limits_{n \to \infty}
        \left(
            \frac{
                \left(\frac{1}{2n}\right)}{
                \left(\frac{1}{n}\right)
            }
        \right)
    ~=~ \frac{1}{10} ~=~ 0,1 < \infty$ \\
$ln ~ n < O(4)$, da $\lim\limits_{n \to \infty}
        \left(
            \frac{(ln ~ n)'}{4'}
        \right)
    ~=~ \lim\limits_{n \to \infty}
        \left(
            \frac{1}{1 \cdot n}
        \right)
    ~=~ 0 < \infty$ \\
$4 \in O(1000)$, da $\lim\limits_{n \to \infty}
        \left(
            \frac{4}{1000}
        \right)
    ~=~ 0,004 < \infty$ \\
$1000 \in O(\sqrt{n}^3)$, da $\lim\limits_{n \to \infty}
    \left(
        \frac{(1000)}{n^\frac{1}{3}}
    \right)
    ~=~ 0 < \infty$\\
$\sqrt{n}^3 \in O(n^\frac{1}{2})$, da $n^\frac{1}{3} < n^\frac{1}{2} ~~ \forall n \in \R \text{ mit $n > 0$}$\\
$n^\frac{1}{2} \in O(n)$, da $n^\frac{1}{2} < n ~~ \forall n \in \R \text{ mit $n > 0$}$\\
$n \in O(n^2)$, da $n < n^2 ~~ \forall n \in \R \text{ mit $n > 0$}$\\
$n^2 \in 2^n$, da 
\begin{math}
    \lim\limits_{n \to \infty}
        \left(
            \frac{(n^2)'}{e^{ln ~ 2 \cdot n}}
        \right)
    ~=~ \lim\limits_{n \to \infty}
        \left(
            \frac{2n}{ln ~ 2 \cdot e^{ln ~ 2 \cdot n}}
        \right)
    ~=~ \lim\limits_{n \to \infty}
        \left(
            \frac{2}{1 \cdot e^{ln ~ 2 \cdot n} + ln ~ 2 \cdot ln ~ 2 \cdot e^{ln ~ 2 \cdot n}}
        \right)\\
    ~=~ \lim\limits_{n \to \infty}
        \left(
            \frac{2}{(1 + (ln ~ 2)^2) e^{ln ~ 2 \cdot n}}
        \right)
    ~=~ \frac{2}{1 + (ln ~ 2)^2} \cdot \lim\limits_{n \to \infty}
        \left(
            \frac{1}{e^(ln ~ 2 \cdot n)}
        \right)
    ~=~ \frac{2}{1 + (ln ~ 2)^2} < \infty
\end{math}
\\
\\
\\
Funktionen gleicher Äquivalenzklassen:\\
\\
$log ~ n \in \Theta(ln ~ n)$, da $log ~ n \in O(ln ~ n)$ (siehe oben) und da\\
\begin{math}
    \lim\limits_{n \to \infty}
    \left(
        \frac{ln ~ n}{log ~ n}
    \right)
    \stackrel{\text{H}}{\Rightarrow} \lim\limits_{n \to \infty}
    \left(
        \frac{2n}{n}
    \right)
    = 2 < \infty \Rightarrow ln ~ n \in O(log ~ n)
\end{math}\\\\
$4 \in \Theta(1000)$, da $4 \in O(1000)$ (siehe oben) und $\lim\limits_{n \to \infty}
    \left(
        \frac{1000}{4}
    \right) = 250 < \infty \Rightarrow 1000 \in O(4)$

\section{}%section name}\label{label}
\begin{math}
    f(n), g(n) \in O(h(n)) \\\\
    \Rightarrow \exists n_0, c ~ (\forall n_{n > n_0} ~ f(n) \leq h(n) \cdot c) 
    \wedge \exists n_0', c' ~ (\forall n_{n > n_0'} ~ g(n) \leq h(n) \cdot c')
    \\
    \Rightarrow \forall n > max(n_0, n_0') ~ (f(n) \cdot g(h) \leq (h(n) \cdot c) \cdot (h(n) \cdot c'))
    \\
    \Rightarrow \forall n > max(n_0, n_0') ~ (f(n) \cdot g(h) \leq h(n) \cdot 2 \cdot c \cdot c')
    \\\\
    \Rightarrow f(n) \cdot g(n) \in O(h(n))
\end{math}

\section{}%section name}\label{label}
\subsection{}%section name}\label{label}
\begin{math}
    T(n) := \left\{ \bfrac{
        0, ~~~~~~~~~ \text{für n = 0}
    }{
        3 \cdot T(n - 1) + 2, ~~ \text{sonst}
    }\right.
\end{math}
\begin{flalign*}
    T(n) &= 3T(n-1) + 2\\
    &= 3(3T(n-2) + 2) + 2 = 9T(n-2) + 6 + 2\\
    &= 9(3T(n-3) + 2) + 6 + 2 = 27 T(n-3) + 18 + 6 + 2\\
    &= ...\\
    &= 3^k T(n-k) + 2 \cdot 3^{k-1} + 2 \cdot 3^{k-2} + ... + 2 \cdot 3^1 + 2 \cdot 3^0 ~~ (*)
\end{flalign*}
\\
(*)\\
\textbf{I) Induktionsanfang:}\
\emph{Behauptung:} $A ~:=~ \exists n ~ (T(n) = 3^k T(n-k) + 2 \cdot 3^{k-1} + ... + 2 \cdot 3^0$\\
\emph{Beweis:} $T(n) = 3T(n-1) + 2 = 3^1 T(n-1) + 2 \cdot 3^0$\\\\
\textbf{II) Induktionsschritt:}\\
\emph{Induktionsvorraussetzung (IV):} $\forall n ~ A(n) \Rightarrow A(n+1)$\\
\emph{Beweis:}\\\\
\begin{math}
    ...
\end{math}

\subsection{}%subsection name}\label{label}

\begin{flalign}
    S(n) := \left\{ \bfrac{
        c, ~~~~~~~~~ \text{für n = 1}
    }{
        16 \cdot S(\frac{n}{4}) + n^2, ~~ \text{sonst}
    }\right.
\end{flalign}
\begin{math}
    (I) ~ log_4(16) = 2\\\\
    (II) ~ n^2 \in \Theta(n^{log_4(16)} = n^2)
    \Rightarrow T(n) \in \Theta(n^{log_4(16)} \cdot log_2(n)) \\
    \Rightarrow T(n) \in \Theta(n^2 \cdot log_2(n))
\end{math}





\end{document}
